\documentclass{article}
\usepackage[polish]{babel}
\usepackage[utf8]{inputenc}
\usepackage{polski}
\frenchspacing
\setcounter{tocdepth}{2}
\usepackage{graphicx}
\graphicspath{ {images/} }
\usepackage{float}
\usepackage{listings}
\usepackage{hyperref}

\usepackage{booktabs}% http://ctan.org/pkg/booktabs
\newcommand{\tabitem}{~~\llap{\textbullet}~~}

\linespread{1.3}
\frenchspacing

\setcounter{tocdepth}{3}

\begin{document}

% -------------------------- TITLEPAGE -------------------------- %

\begin{titlepage}

\newcommand{\HRule}{\rule{\linewidth}{0.5mm}}

\begin{center}

% university
\textsc{\LARGE Politechnika Warszawska}\\[0.5cm]
\textsc{\Large Wydział Matematyki i Nauk Informacyjnych}\\[1cm]

% logo
\includegraphics[width=2cm, height=2cm]{logo}\\[1cm]


\textsc{\Huge Wstęp do algorytmów ewolucyjnych}\\[0.5cm]

%----------------------------------------------------------------------------------------

\HRule \\[0.4cm]
{ \LARGE \bfseries Wielostartowy algorytm wspinaczkowy dla ciągłych zadań optymalizacji o różnych wymiarowościach}\\[0.2cm]
 
%----------------------------------------------------------------------------------------

\HRule \\[0.4cm]
{  \bfseries RAPORT}\\[2.5cm]

% author
\begin{flushright}
	\Large \emph{Autorzy:}\\[0.5cm]
Anna \textsc{Zawadzka}\\
Piotr \textsc{Waszkiewicz}\\[1.5cm]
\end{flushright} 

% date
\vfill
{\large \today}\\[1cm]
	
\end{center}

\end{titlepage}

\newpage
%----------------------------------------------------------------------------------------
\section{Opis problemu}

Celem projektu było zbadanie zachowania wielostartowego algorytmu wspinaczkowego dla ciągłych zadań optymalizacji o różnych wymiarowościach z uwzględnieniem rożnych strategii losowania punktów startowych: losowanie z rozkładem równomiernym, przeszukiwanie po hipersiatce, poisson-disc. Testy przeprowadzone zostały na benchmarku CEC 2013.\\

Podczas działania algorytmu wspinaczkowego można napotkać takie problemy jak: lokalne minima, \textit{plateaux}, czyli równiny oraz wąskie grzbiety. W niektórych problemach mogą pomóc wielokrotne starty z przypadkowych punktów, stąd cel tego zadania - sprawdzenie które podejście może zminimalizować ryzyko nieodnalezienia poszukiwanego globalnego ekstremum.

\section{Metoda realizacji zadania}

\subsection{Algorytm wspinaczkowy}

W projekcie wykorzystany został algorytm wspinaczkowy w wersji z wyborem następnika na podstawie sąsiadów najlepszego znalezionego dotychczas punktu.

\begin{lstlisting}[mathescape][language=R]
x $\leftarrow$ $x_{0}$
while !stop
	y $\leftarrow$ randomNeighbor(x, $\delta$)
	if cec2013(i, y) > cec2013(i, x)
		x $\leftarrow$ y
\end{lstlisting}

Dodatkowo zaimplementowany algorytm co 100 iteracji (czyli co 100 losowań nowego punktu sąsiadującego z najlepszym dotychczas znalezionym rozwiązaniem) zapisuje wartość funkcji celu dla najlepszego osobnika. Warunkiem stopu jest uzyskanie 50 obserwacji.\\

Funkcja \textit{randomNeighbor(x, $\delta$)} zwraca losowy punkt w sąsiedztwie $x$-a, znajdujący się w odległości nie większej niż $\delta$ od $x$-a. Do obliczania odległości stosowana jest metryka euklidesowa. W przypadku wygenerowania punktu, który znajdue się poza dopuszczalnym obszarem, jest on korygowany przy pomocy metody zawijania.\\

Funkcja \textit{cec2013(i, z)} dostępna jest w ramach pakietu CEC2013 i zwraca wartość funkcji $i$ w punkcie $z$. Traktowana jest jako funkcja celu.


\subsection{Strategie losowania punktów startowych}
Punkty startowe wybierane są przy pomocy następujących metod:

\subsubsection{Losowanie z rozkładem równomiernym}
Rozkład równomierny gwarantuje wybranie punktu z określonej n-wymiarowej przestrzeni z jednakowym prawdopodobieństwem. Środowisko R dostarcza funkcję \textit{runif(n, a, b)}, która zwraca n liczb z przedziału [a,b] z rozkładu równomiernego, które traktowane są jako kolejne współrzędne generowanego punktu.

\subsubsection{Przeszukiwanie po hipersiatce}
Metoda ta zakłada istnienie n-wymiarowej siatki. Przecięcia linii tejże siatki wyznaczają kolejne punkty, będące punktami startowymi algorytmu.

\subsubsection{Poisson Disc Sampling}
Technika Poisson Disc generuje zbiór punktów o zadanej liczności w n-wymiarowej przestrzeni, ściśle upakowanych, lecz odległych od siebie co najmniej o podaną minimalną odległość. Algorytm zrealizowany został zgodnie z metodą podaną w poniższym dokumencie:\\
\url{http://www.cs.ubc.ca/~rbridson/docs/bridson-siggraph07-poissondisk.pdf}

\section{Przebieg eksperymentów}

Funkcja cec2013 dostarczona w ramach pakietu CEC2013 przyjmuje dwa argumenty: punkt oraz numer funkcji wykorzystanej do obliczenia wartości funkcji celu dla podanego punktu. Dostępne funkcje numerowane są od 1 do 28, a podany punkt musi być punktem z przestrzeni o wymiarze 2, 5, 10, 20, 30, 40, 50, 60, 70, 80, 90 lub 100. Daje to 336 doświadczeń, które dodatkowo przeprowadzane są dla każdej z trzech strategii losowania punktów startowych, co daje 1008 eksperymentów. Liczba doświadczeń przez nas przeprowadzonych jest jednak mniejsza, ponieważ generowanie punktów metodą Poisson Disc zrealizowane jest tylko dla przestrzeni 2 i 5-wymiarowej ze względu na bardzo dużą złożoność pamięciową.\\
Ponieważ testowana metoda jest niedeterministyczna, każde doświadczenie jest powtarzane 20 razy, a wyniki uśredniane.\\
Ogólna postać głównego algorytmu:
\begin{lstlisting}[mathescape][language=R]
dimensions $\leftarrow$ c(2,5,10,20,30,40,50,60,70,80,90,100);
functions $\leftarrow$ c(1:28);
N $\leftarrow$ 10; 
delta $\leftarrow$ 1e-2;
a $\leftarrow$ -1;
b $\leftarrow$ 1;
foreach d in dimensions
{
    utwórz zbiór N punktów startowych za pomocą losowania z rozkładem równomiernym;
    utwórz zbiór N punktów startowych za pomocą przeszukiwania po hipersiatce;
    utwórz zbiór N punktów startowych za pomocą metody Poisson-Disc;

    foreach f in functions)
    {
        uruchom algorytm wspinaczkowy dla danej funkcji i każdego zbioru punktów;
    }
}
\end{lstlisting}



\section{Stare - koncepcja przeprowadzania eksperymentów}
Pierwszym krokiem doświadczenia będzie wylosowanie N punktów startowych przy pomocy trzech opisanych powyżej metod. Punkty te będą badane dla każdej z funkcji celu z pakietu CEC2013.

Następnym krokiem będzie uruchomienie algorytmu wspinaczkowego dla każdego z początkowych punktów startowych. Co 1000 kroków programu (czyli co 1000 losowań nowego punktu sąsiadującego z najlepszym dotychczas znalezionym rozwiązaniem) wartość funkcji celu dla najlepszego osobnika z populacji będzie notowana. Zakończenie obliczeń jest planowane po uzyskaniu 1000 wyników w trakcie działania programu. Następnie nastąpi wyliczenie wartości średnich dla elementów w wektorze na podstawie wyników otrzymanych dla każdego z punktów startowych. Obliczenia powtórzone będą dla każdej z trzech metod wybierania punktów startowych a otrzymane wyniki zestawione, porównane i opisane. Na koniec zaprezentowane zostaną wyciągnięte wnioski.

\end{document}