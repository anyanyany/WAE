\documentclass{article}
\usepackage[polish]{babel}
\usepackage[utf8]{inputenc}
\usepackage{polski}
\frenchspacing
\setcounter{tocdepth}{2}
\usepackage{graphicx}
\graphicspath{ {images/} }
\usepackage{float}
\usepackage{listings}

\usepackage{booktabs}% http://ctan.org/pkg/booktabs
\newcommand{\tabitem}{~~\llap{\textbullet}~~}

\linespread{1.3}
\frenchspacing

\setcounter{tocdepth}{3}

\begin{document}

% -------------------------- TITLEPAGE -------------------------- %

\begin{titlepage}

\newcommand{\HRule}{\rule{\linewidth}{0.5mm}}

\begin{center}

% university
\textsc{\LARGE Politechnika Warszawska}\\[0.5cm]
\textsc{\Large Wydział Matematyki i Nauk Informacyjnych}\\[1cm]

% logo
\includegraphics[width=2cm, height=2cm]{logo}\\[1cm]


\textsc{\Huge Wstęp do algorytmów ewolucyjnych}\\[0.5cm]

%----------------------------------------------------------------------------------------

\HRule \\[0.4cm]
{ \LARGE \bfseries Wielostartowy algorytm wspinaczkowy dla ciągłych zadań optymalizacji o różnych wymiarowościach}\\[0.2cm]
 
%----------------------------------------------------------------------------------------

\HRule \\[0.4cm]
{  \bfseries Specyfikacja wstępna}\\[2.5cm]

% author
\begin{flushright}
	\Large \emph{Autorzy:}\\[0.5cm]
Anna \textsc{Zawadzka}\\
Piotr \textsc{Waszkiewicz}\\[1.5cm]
\end{flushright} 

% date
\vfill
{\large \today}\\[1cm]
	
\end{center}

\end{titlepage}

\newpage
%----------------------------------------------------------------------------------------
\section{Opis problemu}

Celem projektu jest zbadanie zachowania wielostartowego algorytmu wspinaczkowego dla ciągłych zadań optymalizacji o różnych wymiarowościach z uwzględnieniem rożnych strategii losowania punktów startowych: losowanie z rozkładem równomiernym, przeszukiwanie po hipersiatce, poisson-disc. Testy przeprowadzone będą na benchmarku CEC 2013.\\


%----------------------------------------------------------------------------------------

\section{Metoda realizacji zadania}

\subsection{Algorytm wspinaczkowy}

W projekcie wykorzystany zostanie algorytm wspinaczkowy w wersji z wyborem następnika na podstawie sąsiadów najlepszego, znalezionego dotychczas punktu.

\begin{lstlisting}[mathescape][language=R]
x $\leftarrow$ $x_{0}$
H $\leftarrow$ {$x_{0}$}
while !stop
	y $\leftarrow$ randomNeighbor(x, $\delta$)
	if cec2013(i, y) > cec2013(i, x)
		x $\leftarrow$ y
	H $\leftarrow$ H $\cup$ {y}
\end{lstlisting}
Zbiór H zawiera wszystkie punkty powstałe podczas działania algorytmu i jest nazywany \textit{śladem algorytmu}. Przydaje się podczas graficznej reprezentacji rozłożenia populacji w przestrzeni. W ramach projektu nie będzie on wykorzystywany ze względu na wielowymiarowość problemów, często niemożliwą do narysowania. \\
Funkcja randomNeighbor(x, $\delta$) zwraca losowy punkt w sąsiedztwie x-a, znajdujący się w odległości nie większej niż $\delta$ od x-a. Do obliczania odległości stosowana będzie metryka euklidesowa. \\
Funkcja cec2013(i, z) dostępna jest w ramach pakietu CEC2013 i zwraca wartość funkcji 'i' w punkcie 'z'. Traktowana jako funkcja celu.

Podstawowymi problemami jakie można napotkać podczas działania algorytmu są:
\begin{itemize}
	\item Lokalne minima
	\item Plateaux, czyli równiny
	\item Wąskie grzbiety
\end{itemize}

W niektórych problemach mogą pomóc wielokrotne starty z przypadkowych punktów, stąd cel tego zadania - sprawdzenie które podejście może zminimalizować ryzyko nieodnalezienia globalnego, poszukiwanego ekstremum.


\subsection{Strategie losowania punktów startowych}
\subsubsection{Losowanie z rozkładem równomiernym}
\subsubsection{Przeszukiwanie po hipersiatce}
\subsubsection{Poisson-disc}

\subsection{Koncepcja przeprowadzenia eksperymentu}

Początek doświadczenia - losujemy N punktów startowych \\
Wykonujemy 1000 iteracji i spisujemy wartości najlepszych punktów \\
Takich kroków wykonujemy łącznie 1000 (czyli 1M iteracji) po czym kończymy \\
Porównujemy wyniki dla różnych wersji wyboru punktów startowych \\

\end{document}