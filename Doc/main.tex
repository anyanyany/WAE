\documentclass{article}
\usepackage[polish]{babel}
\usepackage[utf8]{inputenc}
\usepackage{polski}
\frenchspacing
\setcounter{tocdepth}{2}
\usepackage{graphicx}
\graphicspath{ {images/} }
\usepackage{float}
\usepackage{listings}
\usepackage{hyperref}

\usepackage{booktabs}% http://ctan.org/pkg/booktabs
\newcommand{\tabitem}{~~\llap{\textbullet}~~}

\linespread{1.3}
\frenchspacing

\setcounter{tocdepth}{3}

\begin{document}

% -------------------------- TITLEPAGE -------------------------- %

\begin{titlepage}

\newcommand{\HRule}{\rule{\linewidth}{0.5mm}}

\begin{center}

% university
\textsc{\LARGE Politechnika Warszawska}\\[0.5cm]
\textsc{\Large Wydział Matematyki i Nauk Informacyjnych}\\[1cm]

% logo
\includegraphics[width=2cm, height=2cm]{logo}\\[1cm]


\textsc{\Huge Wstęp do algorytmów ewolucyjnych}\\[0.5cm]

%----------------------------------------------------------------------------------------

\HRule \\[0.4cm]
{ \LARGE \bfseries Wielostartowy algorytm wspinaczkowy dla ciągłych zadań optymalizacji o różnych wymiarowościach}\\[0.2cm]
 
%----------------------------------------------------------------------------------------

\HRule \\[0.4cm]
{  \bfseries Specyfikacja wstępna}\\[2.5cm]

% author
\begin{flushright}
	\Large \emph{Autorzy:}\\[0.5cm]
Anna \textsc{Zawadzka}\\
Piotr \textsc{Waszkiewicz}\\[1.5cm]
\end{flushright} 

% date
\vfill
{\large \today}\\[1cm]
	
\end{center}

\end{titlepage}

\newpage
%----------------------------------------------------------------------------------------
\section{Opis problemu}

Celem projektu jest zbadanie zachowania wielostartowego algorytmu wspinaczkowego dla ciągłych zadań optymalizacji o różnych wymiarowościach z uwzględnieniem rożnych strategii losowania punktów startowych: losowanie z rozkładem równomiernym, przeszukiwanie po hipersiatce, poisson-disc. Testy przeprowadzone będą na benchmarku CEC 2013.

\section{Metoda realizacji zadania}

\subsection{Algorytm wspinaczkowy}

W projekcie wykorzystany zostanie algorytm wspinaczkowy w wersji z wyborem następnika na podstawie sąsiadów najlepszego, znalezionego dotychczas punktu.

\begin{lstlisting}[mathescape][language=R]
x $\leftarrow$ $x_{0}$
H $\leftarrow$ {$x_{0}$}
while !stop
	y $\leftarrow$ randomNeighbor(x, $\delta$)
	if cec2013(i, y) > cec2013(i, x)
		x $\leftarrow$ y
	H $\leftarrow$ H $\cup$ {y}
\end{lstlisting}
Zbiór H zawiera wszystkie punkty powstałe podczas działania algorytmu i jest nazywany \textit{śladem algorytmu}. Przydaje się podczas graficznej reprezentacji rozłożenia populacji w przestrzeni. W ramach projektu nie będzie on wykorzystywany ze względu na wielowymiarowość problemów, często niemożliwą do narysowania. \\
Funkcja randomNeighbor(x, $\delta$) zwraca losowy punkt w sąsiedztwie x-a, znajdujący się w odległości nie większej niż $\delta$ od x-a. Do obliczania odległości stosowana będzie metryka euklidesowa. \\
Funkcja cec2013(i, z) dostępna jest w ramach pakietu CEC2013 i zwraca wartość funkcji 'i' w punkcie 'z'. Traktowana jako funkcja celu.

Podstawowymi problemami jakie można napotkać podczas działania algorytmu są:
\begin{itemize}
	\item Lokalne minima
	\item Plateaux, czyli równiny
	\item Wąskie grzbiety
\end{itemize}

W niektórych problemach mogą pomóc wielokrotne starty z przypadkowych punktów, stąd cel tego zadania - sprawdzenie które podejście może zminimalizować ryzyko nieodnalezienia globalnego, poszukiwanego ekstremum.


\subsection{Strategie losowania punktów startowych}
Punkty startowe wybierane będą przy pomocy następujących metod:

\subsubsection{Losowanie z rozkładem równomiernym}
Rozkład równomierny gwarantuje wybranie punktu z określonej n-wymiarowej przestrzeni z jednakowym prawdopodobieństwem. Środowisko R dostarcza funkcję \textit{runif(n, a, b)}, która zwraca n liczb z przedziału [a,b] z rozkładu równomiernego, które traktowane będą jako kolejne współrzędne generowanego punktu.

\subsubsection{Przeszukiwanie po hipersiatce}
Metoda ta zakłada istnienie n-wymiarowej siatki. Przecięcia linii tejże siatki wyznaczają kolejne punkty, będące punktami startowymi algorytmu.

\subsubsection{Poisson Disc Sampling}
Technika Poisson Disc generuje zbiór punktów o zadanej liczności w n-wymiarowej przestrzeni, ściśle upakowanych, lecz odległych od siebie co najmniej o podaną minimalną odległość. Algorytm zrealizowany zostanie zgodny z poniższym dokumentem:\\
\url{http://www.cs.ubc.ca/~rbridson/docs/bridson-siggraph07-poissondisk.pdf}

\section{Koncepcja przeprowadzenia eksperymentu}

Pierwszym krokiem doświadczenia będzie wylosowanie N punktów startowych przy pomocy trzech opisanych powyżej metod. Punkty te będą badane dla każdej z funkcji celu z pakietu CEC2013. \\

Następnym krokiem będzie uruchomienie algorytmu wspinaczkowego dla każdego z początkowych punktów startowych. Co 1000 kroków programu (czyli co 1000 losowań nowego punktu sąsiadującego z najlepszy, dotychczas znalezionym rozwiązaniem) wartość funkcji celu dla najlepszego osobnika z populacji będzie notowana. Zakończenie obliczeń jest planowane po uzyskaniu 1000 wyników w trakcie działania programu. Następnie nastąpi wyliczenie wartości średnich dla elementów w wektorze na podstawie wyników otrzymanych dla każdego z punktów startowych. Obliczenia powtórzone będą dla każdej z trzech metod wybierania punktów startowych a otrzymane wyniki zestawione, porównane i opisane. Na koniec zaprezentowane zostaną wyciągnięte wnioski.

\end{document}